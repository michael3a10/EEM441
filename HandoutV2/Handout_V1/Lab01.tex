\documentclass[11pt,twoside]{SEEELab_V1}

% ---- Metadata ----
\labtitle{Experiment 1 — DC Motor / PID Control}
\labsubtitle{Speed control in open- and closed-loop}
\labauthor{Farhan Aizuddin}
\labdate{September 18, 2025}
\coursecode{EEM441}
\semester{Semester I (2025/2026)}
% Provide a path or comment the next line if not available
\lablogo{Figures/Logo_USM_SEEE.png}

\graphicspath{{Figures/}}

\begin{document}
\makelabtitle

% \SEEEDebugTOCOn % (optional) draw a frame around the margin ToC
\makemargintoc

\section{Introduction}
This experiment covers the basics of DC motor actuation and PID control. You will measure motor speed under open-loop and compare with closed-loop P/PI/PD/PID controllers.

\subsection{Learning Outcomes}
After completing this lab, you should be able to: (i) acquire motor speed data, (ii) design a basic PID controller, and (iii) interpret step responses.

\section{Methodology}
\subsection{Open-Loop Speed Control of a DC Motor}
\begin{enumerate}
  \item Switch on supply and measure reference voltages; calibrate variable voltages.
  \item Disconnect the feedback path and drive the motor with a fixed duty cycle.
  \item Record RPM vs.\ duty cycle and note steady-state behavior.
\end{enumerate}

\subsection{Closed-Loop Control}
\begin{enumerate}
  \item Connect the speed sensor (Hall/encoder) to the controller input.
  \item Tune controllers: P, PD, PI, and PID. Record overshoot, rise time, and settling time.
\end{enumerate}

\section{Results and Discussion}
Present plots of speed vs.\ time for each controller. Discuss stability margins and noise sensitivity.

\section{Conclusion}
Summarize key observations and recommended PID gains for the motor used.
\end{document}
