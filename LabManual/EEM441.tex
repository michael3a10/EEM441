% EEM441_minimal.tex — clean, balanced example
\documentclass[11pt]{SEEELab} % or rename class to SEEELab and use that

% Course/banner info
\coursecode{EEM441}
\semester{Semester I (2025/2026)}
\experimenttitle{DC Motor / PID Control}

% Header logo (single combined)
\lablogo{Figures/Logo_USM_SEEE.png} % comment out if file not available

\begin{document}
\makelabtitle
\makemargintoc

% -------- Margin blocks (lists) --------
\begin{ObjectivesList}
  \item To study open-loop speed control of a DC motor.
  \item To study closed-loop control using:
    \begin{AlphaList}
      \item P Controller
      \item PD Controller
      \item PI Controller
      \item PID Controller
    \end{AlphaList}
\end{ObjectivesList}

\begin{ApparatusList}
  \item Controller kit
  \item Oscilloscope
  \item BNC cables
  \item Multimeter
\end{ApparatusList}


% ------------- Body -------------
\section{Introduction}
This manual outlines the setup and measurement steps for DC motor speed control.
We will compare open- and closed-loop responses and tune PID gains.

\section{Methodology}
\subsection{Open-Loop Test}
Measure supply rails, connect the motor, and record steady-state speed.

\subsection{Closed-Loop Test}
Tune $K_p$, $K_i$, and $K_d$ to minimize error:
\[
u(t) = K_p e(t) + K_i \int_0^t e(\tau)\,d\tau + K_d \frac{de(t)}{dt}.
\]

\appendix
\section*{Appendix}
\begin{lstlisting}
// Example PID loop
error = ref - speed;
u = Kp*error + Ki*I + Kd*(error - error_prev);
apply(u);
\end{lstlisting}

\end{document}
